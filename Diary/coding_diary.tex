%----------------------------------------------------------------------------------------
%	PACKAGES AND DOCUMENT CONFIGURATIONS
%----------------------------------------------------------------------------------------

\documentclass{article}

\usepackage{graphicx} % Required for the inclusion of images
\usepackage{natbib} % Required to change bibliography style to APA
\usepackage{amsmath} % Required for some math elements 

\bibliographystyle{unsrtnat}

\setlength\parindent{0pt} % Removes all indentation from paragraphs

\renewcommand{\labelenumi}{\alph{enumi}.} % Make numbering in the enumerate environment by letter rather than number (e.g. section 6)

%\usepackage{times} % Uncomment to use the Times New Roman font

%----------------------------------------------------------------------------------------
%	DOCUMENT INFORMATION
%----------------------------------------------------------------------------------------


\begin{document}

\begin{Large}
Lappeenrannan teknillinen yliopisto\\
School of Business and Management
\\\\\\\\\\\\\\\\\\\\\
Software Development Skills\\\\
\textbf{Johannes Hölker, Studentnumber: 609983}\\\\\\\\
\textbf{LEARNING DIARY\\ SOFTWARE DEVELOPMENT SKILLS\\ FULL STACK\\MODULE}
\end{Large}

\newpage 


\tableofcontents

\newpage

\section{20.10.2020} 
Dear Diary,\\
After reading the introduction of the course, I replicated the Word Template for the diary in \LaTeX using Texmaker as a GUI for writing. I managed finding my old Github Account then and installed Git on my Ubuntu system. I decided using Fork for my version control because I used it before and I am familiar with the program.
\section{22.10.2020}
I wanted to download Fork but I realised that Fork (and SourceTree) don't have a Linux Version. During watching the Introduction to Version Control I saw that the code editor Atom has an implemented version control GUI, so I decided to take that one.\\
Afterwards I watched the video about Node.js. Because I planned improving my Python skills this semester I realised that the course is made for JavaScript, which I didn't use before so it is a whole new challenge for me. Regarding the point that JavaScript is useful for sending/getting data from a server, I am looking forward to it.\\
I already have an idea for a JS application. With a friend of mine I was planning to make a simple ToDo-list application using the Python. For storing (and checking) the entries of the list on a server, I am thinking about using JavaScript.\\
Now I am setting up my Git-Repository and placing the Texmaker-Workspace, the Exercises Workspace and the ProjectWorkspace in there. During this I am getting familiar with the code editor Atom and the Github implementation in it.

\pagebreak 
%\bibliography{sample}
\end{document}